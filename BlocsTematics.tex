\documentclass{article}
\usepackage[utf8]{inputenc}

%caca pinxada en un pal
\usepackage{aeguill}

\begin{document}

\section{Geometria Projectiva}
\begin{enumerate}
 \item Elements bàsics de geometria projectiva lineal
	\begin{enumerate}
 	\item Espais projectius
 	\item Varietats lineals, incidència
 	\item Projectivitats
 	\item Teoremes de Pappus i Desargues
	\end{enumerate}
 \item Coordenades i raó doble
	\begin{enumerate}
 	\item Referències i coordenades projectives, coordenada absoluta
 	\item Representació de varietats lineals
 	\item Teorema fonamental de la geometria projectiva i representació de projectivitats
 	\item Raó doble, quaterns harmònics
	\end{enumerate}
 \item Geometria afí en el marc projectiu
	\begin{enumerate}
 	\item Clausura projectiva d’un espai afí
 	\item Coordenades afins i projectives
 	\item Varietats lineals d’un espai afí i de la seva clausura
 	\item Afinitats i projectivitats
 	\item Raó simple i raó doble
	\end{enumerate}
 \item Hipersuperfícies quàdriques
	\begin{enumerate}
 	\item Formes multilineals, formes bilineals simètriques i quàdriques d’un espai projectiu
 	\item Conjugació i polaritat
 	\item Elements afins de les quàdriques
 	\item Classificació projectiva, afí i mètrica de quàdriques
	\end{enumerate}
\end{enumerate}
 \newpage

\section{Programació d'Arquitectures Encastades}
\begin{enumerate}
\item Eines de desenvolupament i depuració de sistemes incrustats
	\begin{enumerate}
	\item Entorns d’edició i depuració
	\item Plataformes d’avaluació i depuració
	\end{enumerate}
\item Programació de sistemes encastats
	\begin{enumerate}
	\item Configuració de registres i altres recursos dels processadors
	\item Programació en llenguatge C
	\end{enumerate}
\item Programació d’una plataforma \guillemotleft hardware\guillemotright (robot)% 00bb 00ab
	\begin{enumerate}
	\item Configuració de registres i ports del microcontrolador
	\item Sistema de rellotge i temporitzadors
	\item Estudi del sistema d’interrupcions externes i esdeveniments interns del nostre microcontrolador
	\item Comunicació entre la plataforma del processador i els mòduls del robot
	\item Mesura de sensors i control d’actuadors als mòduls del robot
	\item Control d’un display
	\item Transferències mitjançant accés directe a memòria (DMA)
	\item Altres recursos de la plataforma robot (acceleròmetre, so, etc.)
	\end{enumerate}
\item Tècniques de programació d’arquitectures encastades
	\begin{enumerate}
	\item Programació amb bucles Round-Robin i semàfors
	\item Programació basada en esdeveniments
	\item Abstracció del maquinari (hardware) mitjançant la creació de llibreries de funcions
	\end{enumerate}
\end{enumerate}
\newpage

\section{Sistemes Operatius I}
\begin{enumerate}
\item Introducció
	\begin{enumerate}
	\item Arquitectura dels computadors
	\item Introducció als sistemes operatius
	\end{enumerate}
\item Processos
	\begin{enumerate}
	\item Concepte i estructura
	\item Fil o procés lleuger
	\item Planificació de processos
	\end{enumerate}
\item Gestió de memòria
	\begin{enumerate}
	\item Mapa de memòria
	\item Esquemes d’assignació contigua
	\item Memòria virtual
	\item Polítiques d’assignació i substitució
	\end{enumerate}
\item Llenguatge C
	\begin{enumerate}
	\item Estructures elementals de programació en C
	\item Estructures de dades i memòria dinàmica
	\end{enumerate}
\end{enumerate}

\end{document}
