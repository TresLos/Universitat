\documentclass{article}
\usepackage[utf8]{inputenc}
\usepackage[catalan]{babel}

\usepackage{amsthm, amssymb, amsmath}
\newtheorem{theorem}{Teorema}[section]%chapter
\newtheorem{proposition}[theorem]{Proposició}
\theoremstyle{definition}
\newtheorem{definition}[theorem]{Definició}
\newtheorem{example}[theorem]{Exemple}
\newtheorem{writing}[theorem]{Escriptura}

\newcommand{\R}{\mathbb{R}}
\newcommand{\PP}{\mathbb{P}}

\begin{document}
\tableofcontents
\newpage

\section{Elements bàsics de la geometria projectiva lineal}
\subsection{Espais projectius}
\begin{definition}
Un espai projectiu sobre un cos $K$ (que no tingui característica 2) és una tripleta. $(\R, E, \Pi)$ on $\PP$ és un conjunt, $E$ és un $K$-espai vectorial i $\Pi:E\setminus\{\vec{0}\}\to\PP$
Tal que
\begin{enumerate}
\item $\Pi$ exhaustiva
\item Donat $\vec{u}, \vec{v} \in E\setminus\{0\}$
	$$\Pi (\vec{u}) = \Pi (\vec{v}) \Leftrightarrow <\vec{u}> = <\vec{v}>$$
\end{enumerate}
\end{definition}

\end{document}
