\documentclass{article}
\usepackage[utf8]{inputenc}




\usepackage{dsfont}

\newcommand{\red}[2]{{#1}#2}

\newcommand{\R}{\mathds{R}}

\begin{document}

\section{Tema 1}
Elements bàsics de la geometria projectiva lineal\\
Espai projectiu i varietats lineals

% a que se refereix quan parlem de la característica 2?
\red{Definicio}{Un espai projectiu sobre un cos $K$ (que no estigui característica 2) és una tripleta. $(\R, E, \pi)$ on $\R$ és un conjunt, $E$ és un $K$-espai vectorial i $\pi:E\setminus\{\vec{0}\}\to\mathds{P}$}
Tal que
\begin{enumerate}
\item $\pi$ exhaustiva
\item Donat $\vec{u}, \vec{v} \in E\setminus\{0\}$
\end{enumerate}
$$\pi (\vec{u}) = \pi (\vec{v}) \Leftrightarrow <\vec{u}> = <\vec{v}>$$
(Les antiimatges d'un $p \in \mathds{P}$ són subespai vectorials $\setminus\{0\}$ de dimenció 1)\\

Als elements de $P$ li diem punts. Si tenim $\pi (\vec{u}) = p$ diem que $\vec{u}$ representa $p$.\\

Definim dimensió de $\mathds{P} = dim E -1$\\
Habitualment posem $dim E = n +1$
\begin{itemize}
\item n = 1 Diem $(\mathds{P}, E, \pi)$ recta projectiva
\item n = 2 Diem $(\mathds{P}, E, \pi)$ pla projectiu
\end{itemize}

Abús de notació: Direm "sigui $\mathds{P}$ un espai projectiu..."\\

Exemple 1 $E$ un espai vectoria de dimenció n+1. Definim ($E=\R^3, n=2, K=\R$)\\
$\mathds{P}(E) = \{<u> | u \in E \setminus\{0\}\}$ = \{tots els subespais vectorials de dimenció 1\}\\

Definim $\pi: E\setminus\{0\} \to \mathds{P}(E)$\\
$u \mapsto <u>$\\
$v \mapsto <v>$

\begin{enumerate}
\item És exhaustiu per definició $p \in \mathds{P}(E)$ és un subespai vectorial de dimenció 1 $<u>$, aleshores $\pi(u) = p$
\item $\pi(u) = \pi(v) \Rightarrow <u> = <v>$
\end{enumerate}
Diem que $(\mathds{P}(E), E, \pi)$ és l'espai projectiu associat o el projectivitzat de $E$\\

Exemple 2:

\end{document}
