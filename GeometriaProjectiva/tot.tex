\documentclass{article}
\usepackage[utf8]{inputenc}

\usepackage{mathtools}
% les lletres sobre les fletxes

\usepackage{ upgreek }
% majuscules xD

\usepackage{dsfont}

\newcommand{\red}[2]{{#1}#2}

\newcommand{\R}{\mathds{R}}
\newcommand{\A}{\mathds{A}}

\begin{document}

\section{Tema 1}
Elements bàsics de la geometria projectiva lineal\\
Espai projectiu i varietats lineals

% a que se refereix quan parlem de la característica 2?
\red{Definicio}{Un espai projectiu sobre un cos $K$ (que no tingui característica 2) és una tripleta. $(\R, E, \Pi)$ on $\mathds{P}$ és un conjunt, $E$ és un $K$-espai vectorial i $\Pi:E\setminus\{\vec{0}\}\to\mathds{P}$}
Tal que
\begin{enumerate}
\item $\Pi$ exhaustiva
\item Donat $\vec{u}, \vec{v} \in E\setminus\{0\}$
\end{enumerate}
$$\Pi (\vec{u}) = \Pi (\vec{v}) \Leftrightarrow <\vec{u}> = <\vec{v}>$$
(Les antiimatges d'un $p \in \mathds{P}$ són subespai vectorials $\setminus\{0\}$ de dimenció 1)\\

Als elements de $P$ li diem punts. Si tenim $\Pi (\vec{u}) = p$ diem que $\vec{u}$ representa $p$.\\

Definim dimensió de $\mathds{P} = dim E -1$\\
Habitualment posem $dim E = n +1$ i aleshores $n = dim \mathds{P}$
\begin{itemize}
\item n = 1 Diem $(\mathds{P}, E, \Pi)$ recta projectiva
\item n = 2 Diem $(\mathds{P}, E, \Pi)$ pla projectiu
\end{itemize}

Abús de notació: Direm "sigui $\mathds{P}$ un espai projectiu$\ldots$"\\

Exemple 1 $E$ un espai vectorial de dimenció n+1. Definim ($E=\R^3, n=2, K=\R$)\\
$\mathds{P}(E) = \{<u> | u \in E \setminus\{0\}\}$ = \{tots els subespais vectorials de dimenció 1\}\\

Definim $\Pi: E\setminus\{0\} \to \mathds{P}(E)$\\
$u \mapsto <u>$\\
$v \mapsto <v>$

\begin{enumerate}
\item És exhaustiu per definició $p \in \mathds{P}(E)$ és un subespai vectorial de dimenció 1: $<u>$, aleshores $\Pi(u) = p$
\item $\Pi(u) = \Pi(v) \Rightarrow <u> = <v>$
\end{enumerate}
Diem que $(\mathds{P}(E), E, \Pi)$ és l'espai projectiu associat o el projectivitzat de $E$\\

Exemple 2:\\
$\A^2_k$ pla afí ($\A^2_k, E$, suma)\\
Fixem un punt $O \in \A^2_k$\\
$O^*$ = \{rectes per $O$\}, dimenció $O^*$ = 1\\
eix $E\setminus\{0\} \to O^* = \mathds{P}$\\
$\vec{u} \mapsto O + <\vec{u}>$\\

Definició. Diem que dues estructures projectives $$(\mathds{P}, E, \Pi) i (\mathds{P}, E', \Pi')$$
sobre un mateix conjunt $\mathds{P}$ són equivalents si existeix un isomorfisme
$$\upvarphi: E \xrightarrow{\cong} E'$$
de manera que $E\setminus\{0\} \xrightarrow{\upvarphi} E'\setminus\{0\}$ és commutativa\\
$E\setminus\{0\} \to \mathds{P}\quad E'\setminus\{0\} \to \mathds{P}$\\

Varietat lineal\\
Definició: Una varietat lineal $V$ de $V$ $F\setminus\{0\} \subset E\setminus\{0\} \xrightarrow{F} \mathds{P}$ dimenció $K$ en un espai projectiu $(\mathds{P}, E, \Pi)$ és la imatge per $\Pi$ d'un subespai vectorial $F$ de dimenció $K+1\quad V=\Pi(F\setminus\{0\})$\\
Posem $V = [F]$ per dir que $V$ és la varietat lineal associada a $F$\\

Observació $\Pi(\vec{u}) = p \in \mathds{P}$ punt\\

No ho fem servir\\
$F = <\vec{u}>$\\
$[F] = \{p\}$ varietat lineal de dimenció 0\\

\begin{itemize}
\item dimenció $V$ = 1 $\to V$ és una recta
\item dimenció $V$ = 2 $\to V$ és un plà
\item Codimenció $V$ = 1 $\Leftrightarrow$ dimenció $V$ = n-1 $\to V$ és un hiperplà
\end{itemize}

Codimenció de $V = dim \mathds{P} - dim V$\\

Lema Si $V = [F]$ és una varietat lineal, aleshores $F\setminus\{0\} = \Pi^{-1}(V)$. En particular $V$ només té un subespai associat.\\

Demostració\\
$\subset$) $0 \neq u \in F$ Per definició $\Pi(u) \in \Pi (F\setminus\{0\})$\\
$\supset$) Si $u \in \Pi^{-1}(V) \to \Pi(u) \in V = \Pi(F\setminus\{0\}$\\
$\Rightarrow \exists V \in F\quad \Pi(v) = \Pi(u)$\\
Per b), $<u> = <v> \to u \in F$\\

Propietat de les varietats linelas
\begin{itemize}
\item $L_1 = [F_1] \subset L_2 = [F_2] \leftrightarrow F_1 \subset F_2$
	\subitem $\to) \Pi^{-1}(L_1) \subset \Pi^{-1}(L_2) \leftrightarrow$ apliquem $\Pi$
\item $L_1 \subset L_2 \xRightarrow{F_1 \subset F_2} dim L_1 = dim L_2$
\end{itemize}
Si $L_1 \subset L_2$ i $dim L_1 = dim L_2$, aleshores $L_1 = L_2$

\end{document}
