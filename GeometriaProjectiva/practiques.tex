\documentclass{article}
\usepackage[utf8]{inputenc}
\usepackage[catalan]{babel}

\usepackage{amsthm, amssymb, amsmath, mathtools, MnSymbol}

\newtheorem{resolucio}{Resolució}[section]
\newtheorem{N}{}[section]

\newcommand{\R}{\mathbb{R}}
\newcommand{\PP}{\mathbb{P}}
\newcommand{\U}{\mathcal{U}}

\begin{document}

\section{Suma i intersecció de varietats lineals}
\begin{N}
Sigui $L$ un subconjunt de $\PP^n$. Proveu: $L$ és varietat lineal si i només si $L$ conté la recta $P \vee Q$ per a qualsevol parell de punts diferents $P, Q \in L$.
\end{N}
\begin{resolucio}
\begin{itemize}
\item $L$ subconjunt de $\PP^n$
\item $L$ varietat lineal $\Leftrightarrow \forall p, q \in L$
\item[$\Rightarrow)$] $p, q \in L = [F]$
	\subitem Per la definició de $\vee: \quad p \vee q \subset L$
\item[$\Leftarrow)$] Definim $F = \Pi^{-1}(L) \cup \{0\} = \{u | \Pi(u) \in L\} \cup \{0\}$
	\begin{itemize}
	\item Per construcció $\Pi(F\setminus \{0\} = L$ ``exhaustiu $\Pi\Pi^{-1}(\U) = \U$''
	\item Necesitem veure que $F$ és un subespai. Si ho fem automaticament $L$ serà la varietat lineal associada a $F$.
	\item $u, v \in F \Rightarrow \Pi(u), \Pi(v) \in L \quad p = [<u>], p \vee q = [<u, v>]$
	\item $\Rightarrow p \vee q \subset L$
	\item $\xRightarrow{\Pi^{--1}} <u, v> \subset \Pi^{-1} (L) \cup \{0\} = F$ 
	\item $\Pi(u) = [<u>]$
	\item $\forall u, v \quad <u, v> \subset F \left\{\begin{array}{l}\forall u, v\quad u+v \in F\\\forall u, \lambda \in K,\quad \lambda u \in F\end{array}\right.$
	\end{itemize}
\end{itemize}
\end{resolucio}

\end{document}
