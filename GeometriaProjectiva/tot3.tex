\documentclass{article}
\usepackage[utf8]{inputenc}
\usepackage[catalan]{babel}


\usepackage{amsthm, amssymb, amsmath, mathtools, MnSymbol}
\newtheorem{theorem}{Teorema}[section]%chapter
\newtheorem{proposition}[theorem]{Proposició}
\theoremstyle{definition}
\newtheorem{definition}[theorem]{Definició}
\newtheorem{example}[theorem]{Exemple}
\newtheorem{coroline}[theorem]{Corolari}
\newtheorem{observation}[theorem]{Observació}
\newtheorem{writing}[theorem]{Escriptura}

\newcommand{\R}{\mathbb{R}}
\newcommand{\PP}{\mathbb{P}}

\usepackage{enumitem}

\begin{document}

\begin{definition}[$\PP$]
Espai projectiu $(\PP, E, \Pi)$
\begin{enumerate}
\item $\Pi$ exhaustiu
\item $\Pi(u) = \Pi(v) \Leftrightarrow <u> = <v>$
\end{enumerate}
\end{definition}

$E\setminus\{0\} \xrightarrow{\Pi} \PP \supset L$ varietat lineal
\begin{itemize}
\item $\dim L = \dim F -1$
\item $Codim L = \dim \PP - \dim L$
\item $\dim \PP \coloneqq \dim E -1$
\item $F = \Pi^{-1}(L) \cup \{0\}$, gràcies a que és exhaustiu
	\begin{itemize}
	\item $L_1 = [F_1] \subset L_2 = [F_2] \leftrightarrow F_1 \subset F_2$
	\item $L_1 \subset L_2 \quad \dim L_1 \leq \dim L_2$
	\item $L_1 \subset L_2 \quad \dim L_1 = \dim L_2 \to L_1 = L_2$
	\end{itemize}
\end{itemize}

\begin{observation}
Si considerem $F = (0), \dim F = 0$ podem agafar $L = [F] = \emptyset \to \dim \emptyset = -1$
\end{observation}

\begin{observation}
$\PP = [E]$ varietat lineal total, és l'única varietat de dimensió $n$.\\
Si $L$ té dimensió $\left.\begin{array}{c}n: L \subset \PP\\\dim L = n = \dim \PP\end{array}\right\} L = \PP$
\end{observation}

\begin{observation}
Sigui $L=[F]$ varietat lineal aleshores $\Pi_{|F-\{0\}}:F\setminus \{0\} \to L (= \Pi(F\setminus \{0\})$\\
exhaustiu, $(L, F, \Pi/F\setminus \{0\})$ és un espai projectiu
\end{observation}


\subsection{Incidències de les varietats lineals}
Siguin $L_1 = [F_1]$ i $L_2 = [F_2]$ varietats lineals en un espai projectiu fixat $\PP$
\begin{proposition}
$L_1 \cap L_2$ és una varietat lineal amb subespai associat $F_1 \cap F_2$. És a dir $L_1 \cap L_2 = [F_1 \cap F_2]$
\end{proposition}
\begin{proof}
\begin{itemize}
\item[$\subset$] $[F_1 \cap F_2] \subset L_1 \cap L_2, \quad p \in F_1 \cap F_2$
	\begin{itemize}
	\item $p = \Pi(u)$ tal que $u \in F_1 \cap F_2$ aleshores
		\begin{itemize}
		\item $u \in F_1 \Rightarrow \Pi(u) \in L_1$
		\item $u \in F_2 \Rightarrow \Pi(u) \in L_2$
		\end{itemize}
	\item $p = \Pi(u) \in L_1 \cap L_2$
	\end{itemize}
\item[$\supset$] $p \in L_1 \cap L_2$
	\begin{itemize}
	\item $p \in L_1 \Rightarrow p = \Pi(u_1), u_1 \in F_1$
	\item $p \in L_2 \Rightarrow p = \Pi(u_2), u_2 \in F_2$
	\end{itemize}
\item $\Pi(u_1) = p = \Pi(u_2) \Rightarrow <u_1> = <u_2>$
\end{itemize}
\end{proof}

\begin{definition}
\begin{itemize}
\item $u_1, u_2 \in F_1 \cap F_2$
\item $p \in [F_1 \cap F_2]$
\item $L_1 = [F_1], \ldots, L_r = [F_r]$
\subitem $L_1 \cap \ldots \cap L_r = [F_1 \cap, \ldots, \cap F_r]$
\end{itemize}
\end{definition}

\begin{definition}(generat o joint)
Donat $L_1 = [F_1], L_2 = [F_2]$ varietats lineals. La varietat més petita que conté $L_1 \cup L_2$.\\
La denotarem per $L_1 \vee L_2$
\end{definition}

\begin{proposition}
Es té que $L_1 \vee L_2 = [F_1 + F_2]$
\end{proposition}
\begin{proof}
\begin{itemize}
\item $[F_1 + F_2]$ és una varietat lineal
\item $F_1 \subset F_1 + F_2 \Rightarrow [F_1] \subset [F_1 + F_2]$
\item $F_2 \subset F_1 + F_2 \Rightarrow [F_2] \subset [F_1 + F_2]$
\item Per tant
	\begin{itemize}
	\item $[F_1] \cup [F_2] = L_1 \cup L_2 \subset [F_1 + F_2]$
	\item[Sigui] $M = [G]$ varietat lineal tal que $L_1 \cup L_2 \subset M$
		\begin{itemize}
		\item $\left.\begin{array}{c}
		L_1 \subset M \leftrightarrow F_1 \subset G\\
		L_2 \subset M \leftrightarrow F_2 \subset G\end{array}\right\}\Rightarrow F_1 + F_2 \subset G\quad [F_1 + F_2] \subset M$
		\end{itemize}
	\end{itemize}
\end{itemize}
\end{proof}

\begin{observation}
$L_i = [F_i]$\\$\vee L_i = [\sum F_i]$
\end{observation}

\begin{example}
Si $p, q$ sós dos punts diferents.
\begin{itemize}
\item $p = [u] \neq 0, \quad q = [v] \neq 0$
	\begin{itemize}
	\item[$\leftrightarrow$] $<u> \neq <v>$
	\item[$\leftrightarrow$] $u, v$ són linealment independents
	\item[$\leftrightarrow$] $<u> + <v> = <u, v>$ té dimensió 2
	\end{itemize}
\item $p \vee q = [<u, v>]$ té dimensió 1
\end{itemize}
\end{example}

\begin{theorem}(Fórmula de Grassmann)
\begin{itemize}
\item $L_1, L_2$ varietats lineals en un espai projectiu $\PP$
	\begin{itemize}
	\item $\dim L_1 \vee L_2 + \dim L_1 \cap L_2 = \dim L_1 + \dim L_2$
	\end{itemize}
\end{itemize}
\end{theorem}
\begin{proof}
\begin{itemize}
\item $L_1 = [F_1], L_2 = [F_2]$
	$$(\dim (F_1 + F_2) -1) + (\dim (F_1 \cap F_2) -1) = (\dim F_1 -1) + (\dim F_2 -1)$$
\end{itemize}
\end{proof}

\begin{coroline}
\begin{enumerate}[label=\alph*)]
\item $L$ varietat lineal i $p \notin L$
	\subitem $\dim L \vee p = \dim L + 1$
\begin{proof}
\begin{itemize}
\item $\dim L \cap p = -1$
\item $\dim p = 0$
	\begin{itemize}
	\item $\dim L \vee p + \dim L \cap p = \dim L + \dim p$
	\item $\dim L \vee p = \dim L +1$
	\end{itemize}
\end{itemize}
\end{proof}
\item $\PP$ espai projectiu de dimensió 2
\begin{itemize}
\item $l_1, l_2$ dues rectes $\neq$
\item $l_1 \varsubsetneq l_1 \vee l_2 \Rightarrow l_1 \vee l_2 = \PP^2$
\subitem Dues rectes en un plà diferents, sempre es tallen
\end{itemize}
\begin{proof}
\begin{itemize}
\item $\dim l_1 \vee l_2 = 2$
\item $\dim l_1 = 1$
\item $\dim l_2 = 1$
\item $\dim l_1 \vee l_2 + \dim l_1 \cap l_2 = \dim l_1 + \dim l_2$
	\subitem $2 + \dim l_1 \cap l_2 = 1 + 1 \Rightarrow \dim l_1 \cap l_2 = 0$
\end{itemize}
\end{proof}

\item $L$ varietat lineal. $H$ hiperplà
	\subitem Si $L \nsubset H$ aleshores $\dim L \cap H = \dim L -1$
\begin{proof}
\begin{itemize}
\item Si $L \nsubset H$ aleshores punts $L$ fora de $H$
\item $\PP = H \vee p \subset H \vee L = \PP$
	\subitem $\dim H \vee L + \dim H \cap L = \dim H + \dim L$
	\subitem $\dim H \vee L = \dim L -1$
\end{itemize}
\end{proof}

\item Plà, recta a $\PP^3$
	\subitem o bé recta $\subset$ hiperplà
	\subitem o bé recta $\cap$ hiperplà = punt

\item recta, recta a $\PP^3$, $l_1, l_2$ rectes diferents a $\PP^3$
	\subitem Si tallen $l_1 \cap l_2$ = punt $\to \dim l_1 \vee l_2 = 2$
	\subitem Si no tallen $l_1 \cap l_2 = \emptyset \to \dim l_1 \vee l_2 = 3$ ``creuen''
\end{enumerate}
\end{coroline}

\begin{definition}(suplementaries)
Dues varietats $L_1, L_2$ són suplementaries, si:
\begin{itemize}
\item $L_1 \vee L_2 = \PP$
\item $L_1 \cap L_2 = \emptyset$
\end{itemize}
\end{definition}

\begin{itemize}
\item $[F] \cup [G] = [F + G]$
\item $[F] \cap [G] = [F \cap G]$
\end{itemize}

\end{document}
